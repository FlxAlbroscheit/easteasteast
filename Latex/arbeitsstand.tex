\documentclass{swp}
\usepackage{hyperref}
\usepackage{amsmath}
\usepackage{amssymb}

\begin{document}

\maketitle{Arbeitsstand 15.06.2015}{15.06.2015}{Martin Lechner}
\\\\\\\\\\

\begin{itemize} 
\item Drupal 7 wurde installiert
\item weiterhin implementiert:
\begin{itemize} 
\item Modul \glqq Kalender\grqq{} \url{https://www.drupal.org/project/calendar}
\item Modul \glqq Ctools\grqq{} \url{https://www.drupal.org/project/ctools}
\item Modul \glqq Date\grqq{} \url{https://www.drupal.org/project/date}
\item Modul \glqq View\grqq{} \url{https://www.drupal.org/project/view}
\end{itemize}
\item grobes Layout einer Event-Seite und Akteur-Seite (Ort, Datum, Name, Kurzbeschreibung, Kontakt)
\begin{itemize}
\item Verkn\"upfung mit dem Kalender eingerichtet
\item Beispielinhalte wurden eingetragen 
\end{itemize}
\item Nutzerrollen wurden in Drupal angelegt (Akteur, registrierter User) und entsprechende Berechtigungen gesetzt
\item Kommentarfunktion der Events und Akteure f\"ur registrierte User in Drupal eingerichtet
\item Karte mittels Mapbox aus Open Street Map wird vorbereitet und beinhaltet bereits die Stadtteilgrenzen und Beispielkoordinaten, siehe \url{http://grinch.pavo.uberspace.de/lo/}
\item Entwicklung eines Datenmodells\\\\
\end{itemize}
\end{document}
