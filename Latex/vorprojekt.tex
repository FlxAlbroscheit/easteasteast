\documentclass{swp}
\usepackage{hyperref}
\usepackage{amsmath}
\usepackage{amssymb}

\begin{document}

\maketitle{Vorprojekt}{11.05.2015}{Martin Lechner}
\\\\\\\\\\

\section{Aufgabe}
F\"ur das Vorprojekt soll die Drupal-Plattform bereits in so groben Z\"ugen gebaut werden, dass testweise Datens\"atze von der K.I.L.O.-Website dargestellt werden k\"onnen. Dies umfasst in Bezug auf Drupal ein Kalendermodul sowie Profilseiten f\"ur Akteure. Die \"Uberf\"uhrung der Datens\"atze in das RDF Format wird angestrebt. Hinzu kommt ein erster, grober Entwurf f\"ur das Layout der Seite. Die Vorgehensweise f\"ur die Erstellung von Profilen und Terminen muss nicht der endg\"ultigen Art und Weise entsprechen.
\section{Begr\"undung}
Die Aufgabe umfasst zun\"achst die Einrichtung der f\"ur das Projekt grundlegenden Ressourcen, das hei{ss}t die Installation von Drupal und die Entwicklung einer generellen, funktionsorientierten Struktur und eines entsprechenden Layouts f\"ur die Webseite. Weiterhin greift die Aufgabe bereits zwei zentrale Elemente des Projekts auf: Akteure k\"onnen ein Profil anlegen und Termine erstellen, die im Kalender dargestellt werden. Dar\"uber hinaus k\"onnen externe Daten im RDF Format eingebunden werden. Dadurch wird eine Grundlage f\"ur das weitere Arbeiten mit RDF, und damit f\"ur die Orientierung des Projekts am Konzept Linked Open Data geschaffen.\\

\end{document}
