\documentclass{swp}
\usepackage{hyperref}
\usepackage{amsmath}
\usepackage{amssymb}

\begin{document}

\maketitle{Installationsanleitung}{13.07.2015}{Paul Eisenhuth}
\\\\\\\\\\
Installationsanleitung f\"ur \glqq aae\_{}data\grqq{} Version 1.3 und \glqq aae\_{}theme\grqq{} Version 1.1
\section{Voraussetzungen:}
Eine lauff\"ahige Version von Drupal 7, empfohlen wird Drupal 7.37. Eine Beschreibung der Installation sowie der Download findet sich unter \url{https://www.drupal.org/start}
\section{aae\_{}theme:}
\begin{itemize} 
\item Der Ordner \glqq aae\_{}theme\grqq{} muss in der Drupalinstanz in den Ordner \glqq /drupal/sites/all/themes\grqq{} kopiert werden
\item Auf der Drupalseite einloggen mit dem bei der Installation erstellten Administrator Account und unter \glqq Appearence\grqq{} das Theme \glqq AAE\grqq{} suchen und mit Klick auf \glqq Enable and set default\grqq{} aktivieren
\end{itemize}
\section{aae\_{}data:}
\begin{itemize} 
\item Der Ordner \glqq aae\_{}data\grqq{} muss in der Drupalinstanz in den Ordner \glqq /drupal/sites/all/modules\grqq{} kopiert werden
\item In dem Ordner \glqq aae data\grqq{} in den Unterordner \glqq database\grqq{} wechseln und die PHP-Datei \glqq db\_{}config\grqq{} mit einem Texteditor \"offnen. Hier m\"ussen die Zugangsdaten zur Datenbank eingetragen werden, was wo hin geh\"ort ist durch Kommentare ersichtlich
\item Auf der Drupalseite einloggen mit dem bei der Installation erstellten Administrator Account und unter \glqq Modules\grqq{} die Gruppe \glqq Custom Modules\grqq{} suchen und dort einen Haken bei \glqq AAE Data\grqq{} setzen
\end{itemize}

\end{document}
