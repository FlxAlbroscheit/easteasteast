\documentclass{swp}
\usepackage{hyperref}
\usepackage{amsmath}
\usepackage{amssymb}

\begin{document}

\maketitle{Installationsanleitung}{15.09.2015}{Paul Eisenhuth}
\\\\\\\\\\

\tableofcontents
\newpage
\section{Vorraussetzungen}
Eine lauff\"ahige Version von Drupal 7, empfohlen wird Drupal 7.37. Eine Beschreibung der Installation sowie der Download findet sich unter \url{https://www.drupal.org/start}.
\section{aae\_{}theme}
\subsection{Installation}
\begin{itemize}
\item Der Ordner \glqq aae\_{}theme\grqq{} muss in der Drupalinstanz in den Ordner \glqq /drupal/sites/all/themes\grqq{} kopiert werden.
\item Auf der Drupalseite einloggen mit dem bei der Installation erstellten Administrator Account und unter \glqq Appearence\grqq{} das Theme \glqq AAE\grqq{} suchen und mit Klick auf \glqq Enable and set default\grqq{} aktivieren.
\end{itemize}
\subsection{Anpassung}
Wiederkehrende Elemente wie das Men\"u (im Header und Footer) oder die Short-Info-Leiste (im Footer) werden in Drupal als Block  verwaltet und angezeigt. Diese werden wie im Folgenden angelegt:
\begin{itemize}
\item Als Administrator im Backend \glqq Structure $>$ Blocks\grqq{} \"offnen. Auf \glqq add new block\grqq{} klicken. Eingabe folgender Daten, dann best\"atigen:\\\\
 \begin{tabular}{ll}
  Titel 1 & $<$none$>$\\
   & \\
  Description & Short-Info-Leiste\\
   & \\
  Text (beliebig anpassbar - bitte in HTML & $<$br /$><$p$>$Der \\
   schreiben und Links manuell eintragen) & $<$strong$>$Leipziger Osten$<$/strong$>$\\
   & besteht aus neun Bezirken mit insgesamt \\
   & 205.000 Einwohnern...$<$/p$><$br /$>$\\
 & $<$p$><$strong$>$Weitere Informationen:$<$/strong$><$br /$>$\\
  & $<$a href=\glqq DRUPAL LINK \"UBER UNS\grqq{} $>$\\
  & \"Uber uns$<$/a$>$ \\
  & $<$a href=\glqq DRUPAL LINK FAQ\grqq{}$>$\\
  & FAQ$<$/a$>$ \\
  & $<$a href=\glqq DRUPAL LINK ZUM JOURNAL\grqq{}$>$\\
  & Journal$<$/a$>$ \\
  & $<$a href=\glqq DRUPAL LINK KONTAKT\grqq{}$>$\\
  & Kontakt$<$/a$>$\\
  & $<$/p$>$\\
   & \\
  Region & AAE Theme $>$ Footer\\
 \end{tabular}
 \item Als weiteres ist das Men\"u dran. Im Backend Structure $>$ Blocks \"offnen. W\"ahle \glqq Main menu\grqq{}. Eingabe folgender Daten, dann best\"atigen:\\\\
 \begin{tabular}{ll}
 Titel & $<$none$>$\\
  & \\
 Region & AAE Theme $>$ Navigation\\
 \end{tabular}
 \item Es sollte nun im Frontend ein Men\"u angezeigt werden. Um dieses zu f\"ullen, m\"ussen\\
 (Unter-)Seiten erstellt und im Anschluss dem Men\"u hinzugef\"ugt werden (s. hierzu Handbuch).
\end{itemize}
\subsection{Slider}
Um den Slider auf der Startseite mit Leben zu f\"ullen, sei bitte das anliegende Handbuch zu lesen (Administrator $>$ Slider Settings).
\section{aae\_{}data}
\begin{itemize}
\item Der Ordner \glqq aae\_{}data\grqq{} muss in der Drupalinstanz in den Ordner \glqq /drupal/sites/all/modules\grqq{} kopiert werden.
\item Im Ordner  \glqq aae data\grqq{} in den Unterordner \glqq database\grqq{} wechseln und die PHP-Datei \glqq db\_{}config.php\grqq{} mit einem Texteditor \"offnen. Hier m\"ussen die Zugangsdaten zur Datenbank eingetragen werden; was wo hin geh\"ort ist durch Kommentare ersichtlich.
\item Auf der Drupalseite einloggen mit dem bei der Installation erstellten Administrator Account und unter \glqq Modules\grqq{} die Gruppe \glqq Custom Modules\grqq{} suchen und dort einen Haken bei \glqq AAE Data\grqq{} setzen.
\item Nach der erfolgreichen Installation sollten die URLs .../?q=akteure oder .../?q=events im Browser abrufbar sein. Geht dies noch nicht, Modul einfach neu-installieren und erneut pr\"ufen. Danach via Structure $>$ menues $>$ Main menu einen Link auf diese beiden Seiten manuell anlegen.
\end{itemize}
\section{Texteditor f\"ur Journal}
F\"ur ein komfortableres und leichter zu erlernendes Schreiben des Journals empfehlen wir die Einbindung eines Texteditors mit Hilfe des Moduls \glqq WYSIWYG\grqq{}. Dieses Modul wird von Dritten bereitgestellt und gewartet, daher hier nur der Verweis auf deren Installationsanleitung (\url{https://www.drupal.org/node/371459}) sowie den Download (\url{https://www.drupal.org/project/wysiwyg}).\\
Die Auswahl des konkreten Testeditors ist nur nach pers\"onlichen Vorlieben zu treffen, der \glqq TinyMCE\grqq{}-Editor wird beispielsweise von uns genutzt.\\
Die Aktivierung eines Editors muss auf jeden Fall f\"ur die Input Art \glqq plain text\grqq{} vorgenommen werden, da dieser als einziger den Redakteuren zur Verf\"ugung steht.
\section{Redakteurrolle}
Um den Redakteuren ihre Arbeit zu erm\"oglichen und sie von gew\"ohnlichen eingeloggten Usern zu unterscheiden, muss diese Rolle in Drupal angelegt werden:\\
\begin{itemize}
\item Auf der Drupalseite einloggen mit dem bei der Installation erstellten Administrator Account und unter \glqq People\grqq{} in den Reiter \glqq Permissionsgrqq{} wechseln, dort in den Subreiter \glqq Rolles\grqq{} wechseln und in das Textfeld \glqq redakteur\grqq{} schreiben und mit Klick auf \glqq Add Role\grqq{} best\"atigen.
\item Neben \glqq redakteur\grqq{} auf \glqq edit permissions\grqq{} klicken.
\item Unter \glqq Node\grqq{} einen Haken setzen bei \glqq Access the content overview page\grqq{}, \glqq Article: Create new content\grqq{}, \glqq Article: Edit own Content\grqq{} und \glqq Article: Delete own Content\grqq{}.
\item Unter  \glqq Overlay\grqq{} einen Haken setzen bei \glqq Access the administrative overlay\grqq{}.
\item Mit Klick auf \glqq Save permissions\grqq{} am Ende der Seite best\"atigen.
\end{itemize}
\end{document}
