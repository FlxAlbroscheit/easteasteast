\documentclass{swp}
\usepackage{hyperref}
\usepackage{amsmath}
\usepackage{amssymb}

\begin{document}

\maketitle{Release 29.06.2015}{29.06.2015}{Paul Eisenhuth}
\\\\\\\\\\

\section{Inhalt:}
\begin{itemize} 
\item Drupal Modul \glqq aae\_{}data\grqq{} Version 1.1
\item Drupal Theme \glqq aae\_{}theme\grqq{} Version 1.0
\end{itemize}
\section{\"Anderungen:}
\begin{itemize} 
\item Eine erste Version des \glqq aae\_{}theme\grqq{} wurde erarbeitet und dem Release hinzugef\"ugt.
\item Ein Formular zur Dateneingabe fu\"ur Akteurs-Profile sowie eine Anzeigeseite wurden
eingebunden.
\item Bei Deinstallation und Deaktivierung werden nun die Tabellen gelo\"oscht, anstatt nur bei
der Deaktivierung.
\end{itemize}
\section{Installationsanleitung:}
\subsection{Vorraussetzung:}
Eine lauffa\"ahige Version von Drupal 7, empfohlen wird Drupal 7.37. Eine Downloadbeschreibung der Installation findet sich unter \url{https://www.drupal.org/start}.
\subsection{Installation Theme:}
\begin{enumerate}
\item Der Ordner \glqq aae\_{}theme\grqq{} muss in der Drupalinstanz in den Ordner \glqq ~/drupal/sites/all/themes\grqq{} kopiert werden.
\item  Auf der Drupalseite einloggen mit dem bei der Installation erstellten Administrator Account und unter \glqq Appearence\grqq{} das Theme \glqq AAE\grqq{} suchen und mit Klick auf \glqq Enable and set default\grqq aktivieren.
\end{enumerate}
\subsection{Installation Module:}
\begin{enumerate}
\item Der Ordner \glqq aae\_{}data\grqq{} muss in der Drupalinstanz in den Ordner \glqq ~/drupal/sites/all/modules\grqq{} kopiert werden.
\item In dem Ordner \glqq aae\_{}data\grqq{} in den Unterordner \glqq database\grqq{} wechseln und die PHP-Datei \glqq db\_{}config\grqq{} mit einem Texteditor o\"offnen. Hier mu\"ussen die Zugangsdaten zur Datenbank eingetragen werden, was wo hin geho\"ort ist durch Kommentare ersichtlich.
\item Auf der Drupalseite einloggen mit dem bei der Installation erstellten Administrator Account und unter \glqq Modules\grqq{} die Gruppe \glqq Custom Modules\grqq{} suchen und dort einen Haken bei \glqq AAE Data\grqq setzen.
\end{enumerate}
\section{Funktionen:}
Nach dem Abschluss der Installation werden automatisch 6 Tabellen in der Datenbank angelegt. Diese werden im Lauf der Entwicklung des Moduls beno\"otigt. Wenn das Modul deaktiviert wird, werden diese Tabellen wieder gelo\"oscht. Au{\ss}erdem sollen in der Navigationsleiste am Rand neue Links erscheinen, die zu den erwa\"ahnten Seiten f\"ur Profileingabe und -anzeige fu\"uhren.
\section{Bekannte Probleme:}
Die neuen Links tauchen nicht auf und die neuen Funktionen sind damit nicht verfu\"ugbar. Au{\ss}erdem ist das Theme noch nicht richtig angepasst, sodass es das Layout der Seite eher kaputt macht, als es zu verbessern.\\
Diese Probleme wurden bei einem manuellen Test am 29.06.2015 um 18:30 Uhr und 20:30 Uhr auf zwei verschiedenen Umgebungen festgestellt. Auf einer dritten Umgebung funktioniert um 20:30 Uhr das Modul, wa\"ahrend das Theme-Problem auch hier besteht.

\end{document}
