\documentclass{swp}
\usepackage{hyperref}
\usepackage{amsmath}
\usepackage{amssymb}

\begin{document}

\maketitle{Arbeitsplan}{21.05.2015}{Martin Lechner}
\\\\\\\\\\

\tableofcontents
\newpage
\section{Projektvision}
Es soll eine Stadtteilplattform f\"ur den Leipziger Osten, unter Nutzung von Drupal und Linked Data, erstellt werden. Auf dieser Plattform k\"onnen lokale Akteure des Leipziger Ostens Aktivit\"aten und Angebote pr\"asentieren und sich austauschen. Besucher der Seite k\"onnen sich \"uber Projekte informieren und anhand einer Filterfunktion, sowie einer Karte und einem Kalender Angebote heraussuchen. Die Plattform soll weitestgehend alle Zielgruppen ansprechen und daher einfach, \"ubersichtlich und barrierefrei gestaltet werden. Dazu soll im Rahmen des Praktikums eine funktionsf\"ahige Struktur mit grundlegenden Funktionen aufgebaut werden, die in der Folgezeit St\"uck f\"ur St\"uck erweiterbar ist, so dass in einiger Zeit eine umfassende Kommunikationsbasis auf Stadtteilebene entsteht. Die Plattform soll mit Plattformen anderer Akteure Daten austauschen k\"onnen und sich bei diesem Austausch an den Konzepten des Leipzig Data Projekts orientieren.

\section{Voraussetzungen}
F\"ur die erfolgreiche und p\"unktliche Implementierung aller in der Projektvision und den Arbeitspaketen benannten Funktionen und Schritte bedarf es folgender Notwendigkeiten:
\begin{itemize}
\item Auf technischer Seite m\"ussen ein Web- und FTP-Server, Git-Software, SQL-Datenbank (bspw. MySQL, MariaDB o.\"a.) sowie s\"amtliche Pakete zur Ausf\"uhrung von Server-seitigem Code (in unserem Fall sollte PHP, zur Ausf\"uhrung von Drupal 7, ab Version 5.4.5 inkl. den vorinstallierten Modulen vorliegen) installiert und konfiguriert worden sein. Es muss ausreichend Speicherplatz auf dem Zielsystem zur Verf\"ugung stehen und ein Nutzer eingerichtet worden sein, welcher via Kommandozeile das System verwalten kann (bspw. Hinzuf\"ugen neuer Software oder weiterer Benutzer, Eintragen der SSH-Schl\"ussel,...). Diese Punkte wurden bereits vom URZ realisiert und dem Projekt eine \"offentliche URL zur Verf\"ugung gestellt; nichtsdestotrotz sollte zu Beginn eine reibungslose Interaktion aller Module gepr\"uft werden.
\item Interne Synchronisation: Durch den Team-weiten Einsatz der OLAT-Plattform, Scrum-Methodik (samt entsprechender Dokumentationspflichten) sowie mehrmaliger Treffen pro Woche k\"onnen wir sicherstellen, dass alle Mitglieder ein fortschreitendes Bild \"uber die genauen Einsatzm\"oglichkeiten dieses Projektes gewinnen bzw. sich einen gemeinsamen Wissensstand aneignen.
\item Damit die Bezirke des Leipziger Ostens von dieser Online-Plattform profitieren, bedarf es engagierter, lokaler Akteure, welche Content \"uber Projekte und Veranstaltungen einstellen und pflegen. Eine ausreichende Zahl partizipierender, interessierter (und m\"oglichst wiederkehrender) Nutzer spielt dabei eine ebenso gro{ss}e Rolle. Um beides zu erreichen, sollten die im Konzept vermerkten Alleinstellungsmerkmale unseres Angebotes herausgearbeitet und auf weitere Ideen externer Akteure eingegangen werden.
\end{itemize}

\section{Design\"ubersicht und Funktionalit\"at}
\textbf{Use Case 1: Suchen und Filtern von Veranstaltungen/Angeboten/Akteuren}\\\\
\textbf{Beschreibung:} Der User m\"ochte sich auf der Plattform \"uber Veranstaltungen oder Akteure informieren und die Suche durch Filterung einschr\"anken und verfeinern.\\
\textbf{Funktionale Anforderungen:} Die Plattform muss auf jeder Seite ein Suchfeld zur Verf\"ugung stellen, sowie ein Suchformular bereitstellen, in welchem weitere Suchoptionen ausgew\"ahlt werden k\"onnen. Diese sind: Datum/Zeitraum, Ort, Veranstaltung/Akteur, Mobilit\"atsangaben (Erreichbarkeit mit Bus, Bahn, Zug, Auto, Rad, zu Fu{ss}), Barrierefreiheit (Zugang f\"ur Rollstuhlfahrer, Wickeltisch, ...), Themen/Bereiche, Zielgruppe.\\
Weiterhin soll die Suche mittels einer interaktiven Karte, sowie eines Kalenders m\"oglich sein. Dabei stehen gleichfalls die oben genannten Filteroptionen zur Verf\"ugung.\\
Au{ss}erdem soll auf er Startseite ein Newsfeed \"uber die wichtigsten und aktuellsten Akteure und Veranstaltungen informieren.\\
\textbf{Ziel:} Der User findet Informationen \"uber eine Veranstaltung/einen Akteur.\\
\textbf{Beteiligte Rollen:} alle\\

\textbf{Use Case 2: Kontaktaufnahme mit Akteuren}\\\\
\textbf{Beschreibung: }Der User m\"ochte mit einem Akteur in Kontakt treten.\\
\textbf{Funktionale Anforderungen: }Jeder Akteur bietet mindestens eine Kontaktm\"oglichkeit an. Dies kann sein: Email-Adresse, Telefonnummer oder Verwendung eines Kontaktformulars.\\
\textbf{Ziel: }Der User nimmt erfolgreich Kontakt mit dem Akteur auf.\\
\textbf{Beteiligte Rollen: }alle\\

\textbf{Use Case 3: Registrierung}\\\\
\textbf{Beschreibung: }Ein User m\"ochte sich registrieren.\\
\textbf{Funktionale Anforderungen: }Auf jeder Seite der Plattform muss die M\"oglichkeit zur Registrierung gegeben sein. Der User muss f\"ur die Registrierung einen Nutzernamen und ein Passwort hinterlegen.\\
\textbf{Ziel: }Der User hat sich erfolgreich registriert.\\
\textbf{Beteiligte Rollen: }alle nicht registrierten User\\

\textbf{Use Case 4: Login/Logout}\\\\
\textbf{Beschreibung: }Ein User m\"ochte sich in sein Userkonto einloggen/ aus seinem Userkonto ausloggen.\\
\textbf{Funktionale Anforderungen: }Auf jeder Seite der Plattform muss die M\"oglichkeit zum Einloggen gegeben sein und, wenn bereits eingeloggt, zum Ausloggen. Zum Einloggen muss der User seinen Nutzernamen und sein Passwort eingeben. Damit sich ein User einloggen kann, muss er sich vorher registriert haben (Use Case 3). Zum Ausloggen muss ein Button angeboten werden.\\
\textbf{Ziel: }Der User ist eingeloggt/ausgeloggt.\\
\textbf{Beteiligte Rollen: }alle registrierten User\\

\textbf{Use Case 5: Kommentieren und Bewerten von Veranstaltungen/Angeboten/Akteuren}\\\\
\textbf{Beschreibung: }Der User m\"ochte eine Veranstaltung oder einen Akteur kommentieren oder bewerten.\\
\textbf{Funktionale Anforderungen: }Es muss auf jeder Akteurseite und Veranstaltungsseite eine Kommentar- und Bewertungsfunktion zur Verf\"ugung stehen. Um kommentieren zu k\"onnen, muss sich der User einloggen (Use Case 4).\\
\textbf{Ziel: }Der User hat einen Kommentar oder/und eine Bewertung erfolgreich abgegeben.\\
\textbf{Beteiligte Rollen: }alle registrierten User\\

\textbf{Use Case 6: Anlegen und Gestalten eines Akteurprofils}\\\\
\textbf{Beschreibung: }Ein Akteur soll in die Plattform aufgenommen werden.\\
\textbf{Funktionale Anforderungen: }Dies kann auf zwei Wegen geschehen. Zum einen durch das automatische Dateneinspeisen mittels RDF von bereits existierenden Internetportalen. Bei der automatischen Profilgenerierung fungiert standardm\"a{ss}ig der Useradmin als Akteurinhaber. Zum anderen durch das Anlegen eines Akteurs durch einen registrieren, eingeloggten User. Dieser User ist dann automatisch der Akteurinhaber, also der Administrator des Akteurprofils. Das Akteurprofil ist in drei Abschnitte gegliedert: Allgemeine Informationen, Kommentarbereich und einen freien Darstellungsbereich. In letzterem k\"onnen Text, Bilder und Videos (aus externen Quellen, wie Youtube, ...) eingebunden werden.\\
\textbf{Ziel: }Ein Akteur ist auf der Plattform vertreten.\\
\textbf{Beteiligte Rollen: }registrierte User, Useradmin\\

\textbf{Use Case 7: Akteurmitglied werden}\\\\
\textbf{Beschreibung: }Ein User m\"ochte aktives/initiatives Mitglied eines Akteuers werden.\\
\textbf{Funktionale Anforderungen: }Der Akteuradmin kann registrierte User zu Akteurmitgliedern ernennen und ihnen bestimmte Bearbeitungsrechte f\"ur das Akteurprofil zugestehen.\\
\textbf{Ziel: }User ist Akterumitglied.\\
\textbf{Beteiligte Rollen: }Akteuradmin\\

\textbf{Use Case 8: Anbieten/Bearbeiten/L\"oschen einer Veranstaltung/eines Angebots}\\\\
\textbf{Beschreibung: }Ein Akteur m\"ochte eine Veranstaltung anlegen, bearbeiten oder l\"oschen.\\
\textbf{Funktionale Anforderungen: }Der Akteur, vertreten durch den Akteuradmin und Akteurmitglieder mit entsprechenden Rechten, kann eine Veransatltung anlegen und Informationen zu der Veranstaltung ver\"offentlichen. Diese k\"onnen sp\"ater noch bearbeitet werden. Bei Veranstaltungsablauf oder -absage, kann die Veranstaltung gel\"oscht werden.\\
\textbf{Ziel: }Veranstaltung ist erstellt/bearbeitet/gel\"oscht.\\
\textbf{Beteiligte Rollen: }Akteuradmin, Akteurmitglieder mit entsprechenden Rechten\\

\textbf{Use Case 9: Userverwaltung}\\\\
\textbf{Beschreibung: }Verwaltung von registrierten Usern.\\
\textbf{Funktionale Anforderungen: }Der Useradmin kann registrierte User und Akteure verwalten, d.h. Registrierungen freischalten, Userkonten l\"oschen, etc.\\
\textbf{Ziel: }Verwaltung von Usern und Akteuren.\\
\textbf{Beteiligte Rollen: }Useradmin\\

\textbf{Use Case 10: Bearbeitung der Plattform}\\\\
\textbf{Beschreibung: }Beabeitung des Layouts, des Inhalts und der Struktur der Plattform.\\
\textbf{Funktionale Anforderungen: }Der Moderator muss die M\"oglichkeit haben, Inhalte (dies beinhaltet auch die Moderation von Kommentaren) zu editieren.\\
Der technische Admin kann die Plattform hinsichtlich Layout und Struktur bearbeiten.\\
\textbf{Ziel: }Bearbeitete Plattform.\\
\textbf{Beteiligte Rollen: }Moderator, technische Admin\\

\textbf{Use Case 11: Ver\"offentlichung einer Meldung im Newsfeed}\\\\
\textbf{Beschreibung: }Der Moderator m\"ochte eine Meldung im Newsfeed ver\"offenlichen.\\
\textbf{Funktionale Anforderungen: }Der Moderator ben\"otigt die M\"oglichkeit, Meldungen im Newsfeed der Plattform zu ver\"offentlichen.\\
Au{ss}erdem sollen neu erstellte Veranstaltungen automatisch im Newsfeed erscheinen.\\
\textbf{Ziel: }Eine neue Meldung wird im Newsfeed der Plattform angezeigt.\\
\textbf{Beteiligte Rollen: }Moderator\\

\textbf{Use Case 12: L\"oschen eines Userkontos oder eines Akteurs}\\\\
\textbf{Beschreibung: }Ein User m\"ochte sein Userkonto l\"oschen oder ein Akteur soll gel\"oscht werden.\\
\textbf{Funktionale Anforderungen: }Ein registrierter User muss die M\"oglichkeit haben, jeder Zeit sein Konto l\"oschen zu k\"onnen. Dazu muss eine entsprechende Option in seinen Kontoeinstellungen gegeben werden.\\
Wird ein User gel\"oscht, welcher Admin eines Akteurprofils ist, so gehen die Akteuradminrechte zur\"uck an den Useradmin.\\
Soll ein Akteur gel\"oscht werden, muss dies vom Akteuradmin beim Useradmin beantragt werden. Der Useradmin kann einen Akteur l\"oschen.\\
\textbf{Ziel: }Userkonto/Akteur ist gel\"oscht.\\
\textbf{Beteiligte Rollen: }alle registrierten User\\

\textbf{Nichtfunktionale Anforderungen:}\\\\
Die Plattform sollte mit ihrem Anliegen und ihren M\"oglichkeiten \"ubersichtlich und eiladend vorgestellt werden. Dies k\"onnte auf einer im Hauptmen\"u oder Startseite verlinkten Seite geschehen oder bereits - zu Teilen - auf der Startseite selbst. Die Erstellung einer FAQ-Liste sollte in Betracht gezogen werden.\\\\
Im Akteurverzeichnis wird eine Listenansicht aller Akteure ausgegeben, deren Profil entsprechend verlinkt ist. Die Liste sollte entsprechend den in Use Case 1 genannten Kriterien filterbar sein.\\\\

\section{Arbeitspakete}
\begin{enumerate}
\item 40\%: Vorprojekt - Installtion von Drupal und Modulen - Kalenderfunktion - erste Datenintegration der Kilo-Daten mittels RDF, so dass diese dargestellt werden k\"onnen - RDF Datenstrukturen kl\"aren - Entwicklung eines groben Designs/Layouts (Reiter, Platzhalter f\"ur Funktionen wie die Karte...)
\item 15\% Nutzer- und Datenverwaltung erm\"oglichen
\begin{itemize}
\item[-]System zur Nutzerverwaltung anlegen
\item[-]Konzept f\"ur die Implementierung von OpenData und Kilo Daten
\item[-]Komponententests
\end{itemize}
\item 5\% Erweiterte Funktionen bereitstellen
\begin{itemize}
\item[-]Drupal-Module f\"ur Karte, Newsfeed finden und einbinden
\item[-]Konkretisierung des Layouts
\item[-]Komponententests
\end{itemize}
\item 25\% Verkn\"upfung der Module bzw. Funktionen:
\begin{itemize}
\item[-]Finden / Entwicklung von Schnittstellen zwischen den Modulen
\item[-]Zusammenspiel der Module durch Entwicklung automatisierter Prozesse f\"ur den Informationsaustausch zwischen ihnen herstellen
\item[-]Integrationstests
\item[-]Funktionalit\"aten unter Ber\"ucksichtigung ihrer einfachen Nutzbarkeit in das Layout der Webseite einbinden
\item[-]Systemtest
\end{itemize}
\item 5\% \glqq Designen\grqq{}
\begin{itemize}
\item[-]Endg\"ultige Umsetzung des Layouts
\item[-]evtl. erneuter Systemtest
\end{itemize}
\item 10\%Auslieferung Endprodukt
\begin{itemize}
\item[-]Testen der eingebundenen Funktionalit\"aten und ihres Zusammenspiels
\item[-]Bug-Fixing
\item[-]Last-Minute Changes
\item[-]Feinschliff des Layouts
\item[-]Abnahmetest
\end{itemize}
\item 25\%(ExtraPaket)
\begin{itemize}
\item[-]Konzept f\"ur und Umsetzung eines \glqq Ressourcenpools\grqq{} auf der Plattform
\item[-]Optional: einfache Nutzerprofil, Terminanmeldung etc.
\end{itemize}
\end{enumerate}

\section{Qualit\"atssicherung}
Dokumentation: \\Die Einigung auf ein einheitliches Dokumentationskonzept ist f\"ur die erfolgreiche und termingerechte Entwicklung der Stadtteilplattform unerl\"asslich. Dadurch wird ein gemeinsames Arbeiten erleichtert und garantiert, dass auch bei personellen Ver\"anderungen ein weiters Voranschreiten ohne Probleme m\"oglich ist. Die Dokumentation des Projekts, sowohl in Protokollen der Treffen, als auch in internen Quelltextkommentaren, erfolgt auf Deutsch. Beim Programmieren halten wir uns an die Coding-Standards von Drupal, die entfernt auf den globalen \glqq PEAR Coding Standards\grqq{} basieren. F\"ur eine automatische Code-Dokumentation verwenden wir phpDocumentor. Aus der gr\"undlichen Dokumentation von Projektverlauf und Coding kann sich sp\"ater auch eine Hilfe-/FAQ-Seite auf der Plattform speisen, die auch technisch unversierten Benutzern die Bedienung des Portals erleichtert.\\\\
Testkonzept:\\Bei einem Projekt dieser Gr\"o{ss}e ist es von Vorteil, so viel wie m\"oglich automatisch testen zu lassen. Viele Entwicklungsumgebungen und Sprachen stellen Tests und Testumgebungen zur Verf\"ugung, so auch Drupal. Diese automatischen Test ber\"ucksichtigen aber keine individuellen Fehler, so dass auch manuelle Tests von Zeit zu Zeit n\"otig sind. Das Testkonzept unterteilt sich in vier Testabschnitte. F\"ur unsere Komponententests und Integrationstests benutzen wir das integrierte Drupalmodul \glqq Testing\grqq{}. F\"ur unsere Systemtests benutzen wir das Tool \glqq Selenium\grqq{}. Dieses automatisiert Browseranfragen und gestattet auch Stresstests, da beliebig viele Anfragen auf einmal gestellt werden k\"onnen. Ein Abnahmetest wird am Ende des Projekts von den Product Ownern durchgef\"uhrt.\\\\
Organisatorisches:\\Die Zusammenarbeit wird in regelm\"a{ss}igen Treffen, die zweimal pro Woche stattfinden, koordiniert. Bei diesen Treffen werden weitere Arbeitsschritte geplant, beziehungsweise erledigte vorgestellt. Weitere Kommunikation erfolgt \"uber das OLAT-Forum und private Kan\"ale. F\"ur gemeinsames Programmieren haben wir ein Git-Repository eingerichtet. Jede \"Anderung ist dort zu beschreiben. Dabei ist sich an die vorab genannten Standards zu halten, um unn\"otige Umformatierungs\"anderungen zu vermeiden.

\section{Glossar}
Leipziger Osten:\\Der Leipziger Osten ist definiert durch die Stadtgebiete: Neustadt Neusch\"onefeld, Volkmarsdorf, Anger-Crottendorf, Sellerhausen - St\"unz, Paunsdorf, M\"olkau, Heiterblick, Engelsdorf, Baalsdorf, und Althen- Kleinp\"osna. Zu kl\"aren w\"are, in wie fern man Nordost und S\"udost mit einbezieht (oder erweiterbar macht) oder nicht.\\

Akteure: \\Unter Akteure sind Folgende zu verstehen: Veranstalter, Vereine, Initiativen und Privatpersonen, die im Leipziger Osten auf verschiedene Art aktiv sind.\\

einfache User:\\Personen, die die Plattform nutzen, um sich zu informieren und \"uber die Kommentarfunktion auszutauschen. In Abgrenzung zu Akteuren sind einfache User nicht an der Pr\"asentation ihrerselbst \"uber ein \"offentliches Profil auf der Plattform interessiert.\\

nicht registrierte User:\\Besucher der Stadtteilplattform, welcher kein Userkonto eingerichtet hat.\\

registrierte User:\\Person, die ein Userkonto auf der Stadtteilplattform besitzt.\\

Akteuradmin:\\Registrierter User, der das Profil eines Akteurs verwaltet.\\

Akteurmitglied:\\Registrierter User, der vom Akteuradmin als zus\"atzlicher Profilverwalter mit eingeschr\"ankten Rechten autorisiert wurde.\\

Moderator/Redakteur:\\Administrator f\"ur alle inhaltsbezogenen Verwaltungsaufgaben der Plattform (Kommentarverwaltung, Informationsverwaltung).\\

Useradmin:\\Administrator f\"ur die Userverwaltung der Plattform.\\

technischen Admin:\\Administrator f\"ur alle technischen Aufgaben (Backendverwaltung/Drupal/...) der Plattform.\\

Seiteninhaber:\\Inhaber der Stadtteilplattform.\\
Der Seiteninhaber kann eine Person oder Institution, wie ein Verein, sein. In der Praxis k\"onnen im Seiteninhaber auch die drei Plattformadministratoren vereint sein. Desweiteren ist eine andere Administrationsverteilung denkbar.\\

Veranstaltung:\\Eine Veranstaltung kann von einem Akteur erstellt werden. Diese kann einmalig, regelm\"a{ss}ig oder unregelm\"a{ss}ig stattfinden.\\

Kurzdarstellung: \\Selbstdarstellung von Akteuren auf der Stadtteilplattform, \"ahnlich eines Profils. Dies sollte mindestens folgende Daten umfassen: Name, Beschreibung, Adresse, sonstige Kontaktm\"oglichkeit (E-Mail, Facebook...), Sparte, Zielgruppe, Optional sind Bilder, ..., etc. Um ein Profil auf der Plattform anzulegen, ben\"otigen die Akteure eine entsprechende Zugangsm\"oglichkeit.\\

Stadtteilplattform: \\Eine interaktive (Online)-Plattform, welche der Organisation, Versch\"onerung, Attraktivit\"at, Vermittlung, \glqq News-Verbreitung\grqq{} und vielem mehr dienen soll. Die Plattform sollte so aufgesetzt sein, dass sie in gewisser Weise selbst fuktioniert. D.h. Akteure und Kunden k\"onnen sich registrieren und Programme und Angebote erstellen und aufzeigen, ohne dass alles von einem Betreiber der Seite einzeln kontrolliert werden muss. (Aus Inhaltlichen, Gesetzlichen, Datenschutz bez\"uglichen Gr\"unden). Ziel der Plattform ist es, eine \"ubersichtliche Website zu gestalten die mittels Interaktiver Karte, Kalender, etc. den Stadtteil mit seinen Akteuren attraktiv macht.

\end{document}
