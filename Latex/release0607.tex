\documentclass{swp}
\usepackage{hyperref}
\usepackage{amsmath}
\usepackage{amssymb}

\begin{document}

\maketitle{Release 06.07.2015}{06.07.2015}{Paul Eisenhuth}
\\\\\\\\\\

\section{Inhalt:}
\begin{itemize} 
\item Drupal Modul \glqq aae\_{}data\grqq{} Version 1.2
\item Drupal Theme \glqq aae\_{}theme\grqq{} Version 1.0
\end{itemize}
\section{\"Anderungen:}
\begin{itemize} 
\item Das Datenmodell wurde auf Grund der j\"ungsten Entwicklungen bzgl. der Zusammenarbeit mit Matthias Petzold neu aufgesetzt
\item Ein Formular zur Dateneingabe f\"ur Akteurs-Profile sowie eine Anzeigeseite wurden
an das neue Schema angepasst und eingebunden.
\end{itemize}
\section{Installationsanleitung:}
\subsection{Vorraussetzung:}
Eine lauffa\"ahige Version von Drupal 7, empfohlen wird Drupal 7.37. Eine Downloadbeschreibung der Installation findet sich unter \url{https://www.drupal.org/start}.
\subsection{Installation Theme:}
\begin{enumerate}
\item Der Ordner \glqq aae\_{}theme\grqq{} muss in der Drupalinstanz in den Ordner \glqq ~/drupal/sites/all/themes\grqq{} kopiert werden.
\item  Auf der Drupalseite einloggen mit dem bei der Installation erstellten Administrator Account und unter \glqq Appearence\grqq{} das Theme \glqq AAE\grqq{} suchen und mit Klick auf \glqq Enable and set default\grqq aktivieren.
\end{enumerate}
\subsection{Installation Module:}
\begin{enumerate}
\item Der Ordner \glqq aae\_{}data\grqq{} muss in der Drupalinstanz in den Ordner \glqq ~/drupal/sites/all/modules\grqq{} kopiert werden.
\item In dem Ordner \glqq aae\_{}data\grqq{} in den Unterordner \glqq database\grqq{} wechseln und die PHP-Datei \glqq db\_{}config\grqq{} mit einem Texteditor o\"offnen. Hier mu\"ussen die Zugangsdaten zur Datenbank eingetragen werden, was wo hin geho\"ort ist durch Kommentare ersichtlich.
\item Auf der Drupalseite einloggen mit dem bei der Installation erstellten Administrator Account und unter \glqq Modules\grqq{} die Gruppe \glqq Custom Modules\grqq{} suchen und dort einen Haken bei \glqq AAE Data\grqq setzen.
\end{enumerate}
\section{Funktionen:}
Nach dem Abschluss der Installation werden automatisch 6 Tabellen in der Datenbank angelegt. Diese werden im Lauf der Entwicklung des Moduls ben\"otigt. Wenn das Modul deinstalliert wird, werden diese Tabellen wieder gel\"oscht. Au{\ss}erdem sollen in der Navigationsleiste am Rand neue Links erscheinen, die zu den erw\"ahnten Seiten f\"ur Profileingabe und -anzeige f\"uhren.
\section{Bekannte Probleme:}
Die Links der Akteure, die aufgelistet werden, f\"uhren bisher nirgendwohin. Dies liegt daran, dass die Entwicklung noch nicht an dieser Seite angekommen ist.\\\\
Auch werden Fehler bisher nicht abgefangen, die entstehen, wenn Formularfelder nicht ausgef\"ullt wurden, deren Werte aber laut Datenbank erforderlich sind.
Ansonsten funktioniert alles wie beschrieben.\\\\
Bei drei manuellen Tests am 06.07.2015 um 21:30 Uhr auf einer lokalen Drupalinstanz und der Praktikumsserverinstanz, sowie um 22:30 auf einer weiteren lokalen Instanz  konnte dies best\"atigt werden.

\end{document}
