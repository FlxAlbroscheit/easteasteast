\documentclass{swp}
\usepackage{hyperref}
\usepackage{amsmath}
\usepackage{amssymb}

\begin{document}

\maketitle{Entwurfsbeschreibung Vorprojekt}{26.05.2015}{Martin Lechner}
\\\\\\\\\\

\tableofcontents
\newpage
\section{Allgemeines}
Die Stadtteilplattform Leipziger Osten soll mit umfangreichen Funktionen ausgestattet sein, um gute Nutzbarkeit zu gew\"ahrleisten. Diese grundlegende Struktur der Plattform wird auf Basis von Drupal realisiert und bis zur Vollendung des Projekts mit Modulen erweitert. Dazu z\"ahlen sowohl bereits vorhandene Drupal-Module, als auch selbstentwickelte, welche die einzelnen Funktionalit\"aten der Plattform miteinander verkn\"upfen sollen. Zur Strukturierung der Datens\"atze wird RDF benutzt und auf die bereits vorhandenen Ontologien von Leipzig Data zur\"uckgegriffen. Um externe Daten im RDF Format einzubinden, muss ein Konzept zur Nutzung geeigneter RDF-Ontologien entwickelt werden. Dadurch wird eine Grundlage f\"ur das weitere Arbeiten mit RDF, und damit f\"ur die Orientierung des Projekts am Konzept Linked Open Data geschaffen.\\\\
F\"ur das Vorprojekt soll die Drupal-Plattform bereits in groben Z\"ugen aufgebaut werden, so dass testweise Datens\"atze dargestellt werden k\"onnen. Dies umfasst in Bezug auf Drupal ein Kalendermodul sowie Profilseiten f\"ur Akteure. Die ersten Inhalte sollen mit extrahierten Datens\"atzen aus der bald offline gehenden K.I.L.O-Website eingebunden werden. Es wird angestrebt, die Datens\"atze daf\"ur bereits in RDF zu \"uberf\"uhren. Ein \"offentliches Profil eines Veranstalters und die Darstellung eines Events in einem Kalender sollen das Minimum darstellen. Die Vorgehensweise f\"ur die Erstellung von Profilen und Terminen im Vorprojekt muss jedoch noch nicht der endg\"ultigen Art und Weise entsprechen.

\subsection{Installationsanleitung}
Drupal muss auf einem Webserver installiert werden. Anweisungen hierf\"ur sind unter \\https://www.drupal.org/documentation/install zu finden. F\"ur die \"Ubertragung der Daten sollte das \glqq Backup and Migrate\grqq{} -Modul genutzt werden. Dieses Modul sammelt ein Abbild der Datenbank, sowie der Drupaleinstellungen und Nutzercontent. Dieses Abbild wird in ein Archive verpackt und kann als einzelne Datei in einer neuen Instanz eingelesen werden. \\(https://www.drupal.org/project/backup$\_$migrate)

\subsection{Backupkonzept}
Um ein sicheres Arbeiten zu gew\"ahrleisten, wird nicht auf der Drupalinstanz des Servers gearbeitet. Jedes Teammitglied installiert sich eine eigene Drupalinstanz auf seinem lokalen Arbeitsrechner. Der \glqq core\grqq{} wird dabei nicht ver\"andert, d.h. Originalthemes und -module werden nicht ver\"andert. Sollten wir \"Anderungen vornehmen m\"ussen, dann werden diese in einem abgeleiteten eigenen Modul implementiert, gleiches gilt f\"ur Themes. Die bearbeiteten Module werden \"uber das git-Repository verwaltet und in Drupal verlinkt, so dass Fehlschl\"age r\"uckg\"angig gemacht werden k\"onnen.

\section{Produkt\"ubersicht}
Das Vorprojekt gliedert sich in drei Bereiche: Installation von Drupal, Implementierung der ben\"otigten Module bzw. Layout der Website und die Dateneinpflege.\\ 
Die Installation von Drupal umfasst sowohl die Installation auf dem Server, als auch die Installation auf den lokalen Rechnern, sowie die Erstellung einer daf\"ur hilfreichen Installationsanleitung.\\
F\"ur das Aufsetzen der Website mit den bereits genannten Funktionen (Kalender, Akteurprofil) m\"ussen Module implementiert werden. Hierbei kann auf vorhandene Module zur\"uckgegriffen werden, welche entsprechend angepasst werden m\"ussen. F\"ur die Darstellung muss ein Layout entwickelt und realisiert werden.\\
Die Dateneinpflege umfasst das Einbinden externer Daten und sp\"ater das Einstellen neuer Inhalte.

\subsection{Drupal}
Die Webseite, die mit Drupal aufgesetzt wurde, soll ein erstes Gesicht bekommen. Elemente, die im Vorprojekt bearbeitet werden, sollen dargestellt werden, ebenso wie sp\"atere Funktionen, wie der Karte. Die Position dieser Funktionen werden gegebenenfalls durch Platzhalter kenntlich gemacht.

\subsection{Layout}
\subsubsection{Darstellung \"offentliches Profil}
Das \"offentliche Profil eines Veranstalters beinhaltet Name, Adresse, E-Mail, Webseite, eine Beschreibung und eventuell ein Bild. Diese Daten werden \"ubersichtlich dargestellt. Weiterhin soll ein Verweis auf ein vom Akteur angebotenes Event eingebunden werden.
\subsubsection{Kalender}
In einem Kalendermodul sollen testweise Events erstellt werden, die mit Akteuren verkn\"upft sind.
\subsubsection{Events}
Event-Seiten sollen via Kalender erreichbar sein und folgende Punkte enthalten: Akteure, Name, Datum, Uhrzeit, Ort, Art der Veranstaltung, Zielgruppe...

\subsection{Inhalte}
Die ersten Inhalte sollen mit extrahierten Datens\"atzen aus der bald offline gehenden K.I.L.O-Website eingebunden werden. Es wird angestrebt, die Datens\"atze daf\"ur bereits in RDF zu \"uberf\"uhren. Die bezogenen Daten umfassen grundlegende Informationen zu den Akteuren.

\section{Grunds\"atzliche Struktur- und Entwurfsprinzipien}
\subsection{MVC-Modell}
F\"ur das Projekt adaptieren wir das MVC-Modell. Im Model werden die Datens\"atze verwaltet. Der Controller reagiert dann auf Nutzeranfragen, wie einer Suche und leitet diese an das Model weiter. Das Model verarbeitet diese Anfrage und gibt das Ergebnis zur Darstellung an den View weiter.

\section{Struktur- und Entwurfsprinzipien einzelner Pakete}
\subsection{Datenebene}
Die Daten stehen im RDF-Format zur Verf\"ugung, bzw. sollen in dieses \"uberf\"uhrt werden. Die Strukturierung orientiert sich an den Ontologien von Leipzig Data.
\subsection{Darstellungsebene}
Die Profilseite eines Akteures und einer Eventseite sollen im Vorprojekt beispielhaft gelayoutet werden. Ein Bereich f\"ur Kommentare und Bewertungen sowie f\"ur eine Kartendarstellung und weitere Bilder werden erst im Hauptprojekt realisiert und sollten im Layout des Vorprojekts durch Platzhalter kenntlich gemacht werden. Selbiges gilt f\"ur allgemeine Layout- und Navigationselemente der Plattform, wie beispielsweise Men\"us, Suchfelder, Buttons, Logo, Schriftz\"uge und den Kopfbereich der Seite.

\section{Datenmodell}
Das Projekt orientiert sich an der RDF-Ontologie des Leipzig Data Projektes und versucht die bestehenden K.I.L.O.-Datens\"atze in diese zu \"ubertragen.

\section{Testkonzept}
\subsection{Komponententests}
Bei der Implementierung einzelner Komponenten (Module) ist es wichtig, diese auf Funktionalit\"at und Sicherheit zu testen. Auch wenn wir viel mit bereits vorhandenen Drupalmodulen realisieren werden, sollte jedes einzelne Modul noch einmal gepr\"uft werden. Nur dadurch ist ein einwandfreies Funktionieren gesichert. Tests werden in Testprotokollen dokumentiert, um allen Output festzuhalten und \"Anderungen nachvollziehen zu k\"onnen. F\"ur die Komponententests benutzen wir das interne Drupalmodul \glqq Testing\grqq{}. \\
Im Rahmen des Vorprojekt wurden bisher noch keine Komponententests durchgef\"uhrt.
\subsection{Integrationstests}
Auch wenn alle Komponenten richtig funktionieren kann es zu Schwierigkeiten beim Zusammenwirken kommen. Es ist daher wichtig, bei dem Einf\"ugen einer Komponente deren Zusammenspiel mit bereits implementierten Komponenten zu testen. Dies wird ebenfalls protokolliert und mittels \glqq Testing\grqq{} gepr\"uft.\\
Im Rahmen des Vorprojekt wurden bisher noch keine Integrationstests durchgef\"uhrt.
\subsection{Systemtests}
Nach allen Komponententests und Integrationstests muss abschlie{ss}end das Gesamtsystem gepr\"uft werden. Es wird getestet, ob das Produkt innerhalb der sp\"ateren Nutzungsumgebung funktioniert. F\"ur immer gleiche Anfragen benutzen wir das Tool \glqq Selenium\grqq{}. Dieses automatisiert Browseranfragen und gestattet auch Stresstests, da beliebig viele Anfragen auf einmal gestellt werden k\"onnen. Auch das wird protokollarisch festgehalten.\\
Im Rahmen des Vorprojekt wurden bisher noch keine Systemtests durchgef\"uhrt.

\section{Glossar}
Leipziger Osten:\\Der Leipziger Osten ist definiert durch die Stadtgebiete: Neustadt Neusch\"onefeld, Volkmarsdorf, Anger-Crottendorf, Sellerhausen - St\"unz, Paunsdorf, M\"olkau, Heiterblick, Engelsdorf, Baalsdorf, und Althen- Kleinp\"osna. Zu kl\"aren w\"are, in wie fern man Nordost und S\"udost mit einbezieht (oder erweiterbar macht) oder nicht.\\

Akteure: \\Unter Akteure sind Folgende zu verstehen: Veranstalter, Vereine, Initiativen und Privatpersonen, die im Leipziger Osten auf verschiedene Art aktiv sind.\\

Veranstaltung:\\Eine Veranstaltung kann von einem Akteur erstellt werden. Diese kann einmalig, regelm\"a{ss}ig oder unregelm\"a{ss}ig stattfinden.\\

Kurzdarstellung: \\Selbstdarstellung von Akteuren auf der Stadtteilplattform, \"ahnlich eines Profils. Dies sollte mindestens folgende Daten umfassen: Name, Beschreibung, Adresse, sonstige Kontaktm\"oglichkeit (E-Mail, Facebook...), Sparte, Zielgruppe, Optional sind Bilder, ..., etc. Um ein Profil auf der Plattform anzulegen, ben\"otigen die Akteure eine entsprechende Zugangsm\"oglichkeit.\\

Stadtteilplattform: \\Eine interaktive (Online)-Plattform, welche der Organisation, Versch\"onerung, Attraktivit\"at, Vermittlung, \glqq News-Verbreitung\grqq{} und vielem mehr dienen soll. Die Plattform sollte so aufgesetzt sein, dass sie in gewisser Weise selbst fuktioniert. D.h. Akteure und Kunden k\"onnen sich registrieren und Programme und Angebote erstellen und aufzeigen, ohne dass alles von einem Betreiber der Seite einzeln kontrolliert werden muss. (Aus Inhaltlichen, Gesetzlichen, Datenschutz bez\"uglichen Gr\"unden). Ziel der Plattform ist es, eine \"ubersichtliche Website zu gestalten die mittels Interaktiver Karte, Kalender, etc. den Stadtteil mit seinen Akteuren attraktiv macht.

\end{document}
