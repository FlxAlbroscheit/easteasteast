\documentclass{swp}
\usepackage{hyperref}
\usepackage{amsmath}
\usepackage{amssymb}

\begin{document}

\maketitle{Handbuch}{15.09.2015}{Paul Eisenhuth}
\\\\\\\\\\

\tableofcontents
\newpage
\section{User}
User, also eingeloggte Besucher der Seite, k\"onnen neue Akteure und Veranstaltungen erstellen, editieren und l\"oschen.
\subsection{Akteure erstellen}
Als eingeloggter User erscheint bei Klick auf \glqq Akteure\grqq{} rechts oben der Button \glqq + Akteur hinzuf\"ugen\grqq{}. Dar\"uber wird das Akteursformular erreicht in dem alle relevanten Daten einzutragen sind. Name des Akteurs und Emailadresse, sowie Bezirk sind Pflicht!
\subsection{Akteur editieren}
Betrachtet man ein Akteursprofil (erreichbar \"uber die Akteursliste), befindet sich im grau-transparenten Menu ein Bearbeitenbutton, vorausgesetzt, dass die n\"otigen Schreibrechte vorhanden sind. \"Uber diesen kann man einen bestehenden Akteur \"andern oder weitere Angaben hinzuf\"ugen. Schreibrechte haben standardm\"a{\ss}ig nur der Administrator, alle Redakteure und der Ersteller eines Akteurs.
\subsection{Akteur l\"oschen}
Betrachtet man ein Akteursprofil (erreichbar über die Akteursliste), befindet sich im grau-transparenten Menu ein Bearbeitenbutton, vorausgesetzt, dass die n\"otigen Schreibrechte vorhanden sind. \"Uber diesen kann man einen bestehenden Akteur l\"oschen, mit Klick auf den L\"oschenbutton.
\subsection{Veranstaltung erstellen}
Als eingeloggter User erscheint bei Klick auf \glqq Events\grqq{} rechts oben ein Button \glqq + Event hinzuf\"ugen\grqq{}. Dar\"uber wird das Eventformular erreicht in dem alle relevanten Daten einzutragen sind. Name des Events, der Veranstalter (Akteur) und das Datum sind Pflicht!
\subsection{Veranstaltung editieren}
Betrachtet man ein Eventprofil (erreichbar über die Eventliste), befindet sich rechts unten ein Editierbutton, vorausgesetzt, dass die n\"otigen Schreibrechte vorhanden sind. \"Uber diesen kann man ein bestehendes Event \"andern oder weitere Angaben hinzuf\"ugen.
Schreibrechte haben standardm\"a{\ss}ig nur der Administrator, alle Redakteure und der Ersteller eines Akteurs.
\subsection{Veranstaltung l\"oschen}
Betrachtet man ein Eventprofil (erreichbar über die Eventliste), befindet sich rechts unten ein L\"oschbutton, vorausgesetzt, dass die n\"otigen Schreibrechte vorhanden sind. \"Uber diesen kann man ein bestehendes Event l\"oschen.

\section{Redakteur}
Ein eingeloggter Besucher, der die Rolle eines Akteurs innehat kann zus\"atzlich Artikel für das Journal schreiben. Au{\ss}erdem kann ein Redakteur, genauso wie der Administrator, Akteure editieren.
\subsection{Artikel schreiben}
Einen neuen Artikel beginnt man mit dem Link \glqq Inhalt hinzuf\"ugen\grqq{} und \glqq Artikel\grqq{}. Dieser erh\"alt einen Titel und kann beliebige \glqq Tags\grqq{} erhalten, nach denen sortiert werden kann. Wenn, wie in der Installationsanleitung empfohlen, ein Editor richtig eingebunden wurde, kann ein Text wie gewohnt geschrieben und formatiert werden. Zum Schluss l\"asst sich noch ein Titelbild f\"ur den Artikel hochladen.\\
Danach folgen einige Einstellungen:\\\\
\glqq Menu Settings\grqq{}:\\
Erzeugt einen Men\"ulink, wird aber nicht ben\"otigt (da Blog) und sollte daher deaktiviert sein.\\\\
\glqq Revision information\grqq{}:\\
Hier k\"onnen Anmerkungen f\"ur andere Redakteure hinterlassen werden.\\\\
\glqq URL path settings\grqq{}:\\
Falls gew\"unscht, kann hier eine eigener Link benannt werden auf den Artikel, wird in der Regel aber nicht ben\"otigt.\\\\
\glqq Comment Settings\grqq{}:\\
Kommentarfunktion aktivieren/deaktivieren.\\\\
\glqq Authoring information\grqq{}:\\
M\"oglichkeit, den eigenen Namen sowie Datum und Uhrzeit einzutragen.\\\\
\glqq Publishing Options\grqq{}:\\
\glqq Published\grqq{} (andere k\"onnen den Artikel lesen), \glqq Promoted to front page\grqq{} (der Artikel erscheint auf der Startseite) und \glqq Sticky at top of lists\grqq{} (Artikel wird immer als erstes aufgelistet).\\\\
Mit Klick auf \glqq Save\grqq{} wird die Arbeit beendet.\\
Unter \glqq Inhalte suchen\grqq{} kann der eigene Artikel mit einem Klick auf \glqq edit\grqq{} bearbeitet werden.
\section{Administrator}
Ein Administrator kann neben s\"amtlichen bereits erw\"ahnten T\"atigkeiten User freischalten und ihnen Rollen geben, sowie Einstellungen am Slider vornehmen und Daten sichern.
\subsection{User Freischalten/Editieren und Rollen vergeben}
Als Administrator erh\"alt man an die hinterlegte Emailadresse Benachrichtigungen, wenn sich User registrieren. Au{\ss}erdem findet man unter \glqq People\grqq{} eine \"Ubersicht aller User. Einen neuen User erkennt man daran, dass er den \glqq Status\grqq{} \glqq blocked\grqq{} und \glqq Last Access\grqq{} \glqq never\grqq{} hat. Unter \glqq Member For\grqq{} kann man lesen, wie lange bereits auf eine Freischaltung gewartet wird.\\\\
\"Uber \glqq edit\grqq{} kann der User unter \glqq Status\grqq{} von \glqq Blocked\grqq{} auf \glqq Active\grqq{} gesetzt werden, daraufhin erh\"alt der User eine Email mit seinem Passwort.\\
Auf dieser Seite lassen sich auch alle weiteren Informationen editieren, dies kann aber auch 	vom User selber vorgenommen werden, mit Ausnahme von \glqq Status\grqq{} und \glqq Roles\grqq{}.\\
Unter \glqq Roles\grqq{} kann mit einem H\"akchen bei \glqq redakteur\grqq{} einem User diese Rolle zugewiesen werden.
\subsection{(Unter-)Seite erstellen}
Als Administrator im Backend Content $>$ Basis Page ausw\"ahlen. Die Eingabe des Titels und des Textes erkl\"art sich von selbst. Im Anschluss sind weiter unten Seiten-interne (\glqq meta-\grqq{})Einstellungen m\"oglich. F\"ur eine Auflistung dieser inkl. Funktion s.o. \glqq Artikelschreiben\grqq{}.\\\\
Wichtig ist in jedem Fall eine Zuteilung der Seite ins Men\"u: Daf\"ur unter \glqq menu settings\grqq{} ein H\"ackchen bei \glqq Provide a menu link\grqq{} setzen, den anzuzeigenden Titel eintragen und dann parent item bei \glqq $<$main menu$>$\grqq{} belassen (Seite erscheint als Men\"ueintrag) oder einem bestehenden Link/Seite unterordnen (Seite erscheint dann zur Auswahl beim \"ubergleiten des Men\"upunktes mit der Maus).\\\\
Problemfall: Obwohl ich meine Seite einem Men\"upunkt untergeordnet habe, wird sie nicht (als Unterseite) im Frontend angezeigt.\\
In diesem Fall m\"ussen die Men\"ueinstellungen angepasst werden. Als Administrator Structure $>$ Menues \"offnen. Beim Men\"u \glqq Main menu\grqq{} auf \glqq list links\grqq{} klicken. Den Link mit der/den entsprechenden problematischen Unterseite(n) heraussuchen und auf \glqq edit\grqq{} klicken. Ein H\"akchen bei \glqq Show as expanded\grqq{} setzen.
\subsection{Daten sichern}
Eine regelm\"a{\ss}ige Datensicherung ist notwendig und unverzichtbar. Neben der M\"oglichkeit, die Datenbank und Drupal durch klassische Abbilder und Kopien zu sichern, findet sich unter \url{https://www.drupal.org/node/22281} eine ausf\"uhrliche Anleitung.
\subsection{Slider-Einstellungen}
Der Slider auf der Hauptseite ist beliebig erweiter- bzw. anpassbar. Dies erfolgt auf manuellem Wege, also nicht via Drupal Backend.\\\\
Mittels Konsole oder durch hoch- und runterladen via FTP-Client ist die Datei \\aae\_{}theme/templates/slider\_{}settings.php zu \"offnen (Im Ordner \glqq sites/default/themes\grqq{}). Dort ist eine n\"ahere Erkl\"arung vermerkt - wichtig ist, dass für jeden \$sliderx-(Array)-Unterpunkt ein Slider-K\"astchen konstruiert wird. Dessen Hintergrund kann unter \glqq image\grqq{} deklariert werden, wobei der Dateiname einem bereits hochgeladenen Bild im aae\_{}theme/img - Ordner entspricht. 

\end{document}
